\documentclass{beamer}
\setbeamertemplate{navigation symbols}{}
\setbeamertemplate{caption}{\insertcaption}
\usetheme{Dresden}
\usepackage{amsmath}
\usecolortheme{crane}


\beamersetuncovermixins{\opaqueness<1>{25}}{\opaqueness<2->{15}}

\begin{document}
	\title{Stats for Data Science}  
	\author{Saumya Bhatnagar}
	\date{\today} 
	
	
\begin{frame}
\titlepage
\end{frame}

\begin{frame}\frametitle{Table of contents}\tableofcontents
\end{frame} 





\section{Regression, Classification, Clustering}
\begin{frame}\frametitle{Regression, Classification, Clustering}
\begin{columns}
	\begin{column}{0.3\textwidth}
		\textbf{Regression}
		\begin{enumerate}
			\item Linear
			\item KNN
			\item SVM
			\item Random Forest
		\end{enumerate}
	\end{column}
	\begin{column}{0.3\textwidth}
		\textbf{Classification}
		\begin{enumerate}
			\item Logistic
			\item KNN
			\item SVM Classifier
			\item Random Forest
		\end{enumerate}
	\end{column}
	\begin{column}{0.3\textwidth}
		\textbf{Clustering}
		\begin{enumerate}
			\item K-Means
			\item Hierarchical
			\item DBSCAN
			\item HDBSCAN
		\end{enumerate}
	\end{column}
\end{columns}
\end{frame}



\section{Regression}
\begin{frame}
	Regression analysis is a statistical technique to assess the relationship between an predictor variable and one or more response factors.
\end{frame}



\begin{frame}[plain]%\frametitle{}
\begin{table}[h]
	\centering
	\begin{tabular}{cccc}
	\textbf{Outcome} & \textbf{GLM Family} & \textbf{Link} & \textbf{Mean to} \\
	\textbf{Variable} & & & \textbf{Variance} \\ 
	\hline %\pause 
	Continuous, & Normal or & \\ unbounded & Standard Gaussian & Identity &  \\  
	\hline
	Continuous, & Gamma or & \\ non-negative & inverse Gamma &  &  \\ \hline
	
	
	Discrete/ & Poisson & Log & Identity \\
	counts/ & Quassi-poisson or &  & If not \\
	rate & negative binomial & & Identity \\
	
	
	\hline
	
	Count & Gamma &  & Over dispersion \\ \hline
	Counts with & Zero inflated poisson & \\ multiple zero & may be checked 
	for fitting & & \\ \hline
	Binary & Binomial or & \\  & Logistic regression & & \\ \hline
	Nominal  & Multinomial regression & \\
	\hline
	\end{tabular} 
\caption{Regression Model Selection Criteria}
\end{table}
\end{frame}



\section{Classification}
\begin{frame}
	Three methods to classifier
	\begin{enumerate}
		\item model a classification rule - knn, decision tree, perceptron, svm
		\item model the probability of class membership given input data - perceptron with cross-entropy cost
		\item make a probabilistic model of data within each class - naive bayes
		1 \& 2 are discriminative classifications
		3 is generative classification
		2 \& 3 probabilistic classification
	\end{enumerate}
\end{frame}

\section{Clustering}
\subsection{Monothetic vs pulythetic}
\begin{frame}
	“Help me understand our customers better so that we can market our products to them in a better manner!\\
	Monothetic: Cluster members have some common property\\
	Polythetic: Cluster members are similar to each other. Distance between elements define relationship\\
\end{frame}

\subsection{K means}

\subsection{hierarchical}

\subsection{dbscan}


\subsection{Analysis}

\begin{frame}\frametitle{Life Time Value (LTV)}
	content...
\end{frame}
\begin{frame}\frametitle{Propensity of Cross-sell}
content...
\end{frame}

\begin{frame}
	Thank You!
\end{frame}

\end{document}